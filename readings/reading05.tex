\documentclass{homework}
\input{preamble}
\author{Jim Fowler}
\course{Math 7851}
\date{Week 5: Cohomology of Grassmannian}

\begin{document}
\maketitle

Last week involved a bit of point-set topology (paracompactness!
normal spaces!) resulting in our identifying bundles over a space $X$
with maps from $X$ into the Grassmannian.  Moreover, identical bundles
result in \textit{homotopic} maps.  This means that, given a bundle
over $X$ and some class $\omega \in H^\star(\mbox{Grassmannian})$, we
can pull it back to get a well-defined class in $H^\star(X)$.  These
\textbf{characteristic classes} are exactly what we have been wanting!
So we had better get started in understanding the cohomology of the
Grassmannian.  And to do that, we'll use a cell structure.

A cell structure of the Grassmannian is exactly what you'll find when
you read Chapter~6 from Milnor--Stasheff's \textit{Characteristic
  Classes}.  That chapter introduces the \textbf{Schubert
  cells}. These cells represent a specific decomposition of the
Grassmannian with applications outside this course. You might already
be familiar with \textbf{flags}; the Schubert cells are certain
subsets of the Grassmannian defined by their incidence with a given
flag.  The combinatorics of such gadgets is important in algebraic
geometry.

\end{document}
