\documentclass{homework}
\usepackage{amsmath}
\DeclareMathOperator{\Mat}{Mat}
\DeclareMathOperator{\End}{End}
\DeclareMathOperator{\Hom}{Hom}
\DeclareMathOperator{\id}{id}
\DeclareMathOperator{\image}{im}
\DeclareMathOperator{\rank}{rank}
\DeclareMathOperator{\nullity}{nullity}
\DeclareMathOperator{\trace}{tr}
\DeclareMathOperator{\Spec}{Spec}
\DeclareMathOperator{\Sym}{Sym}
\DeclareMathOperator{\pf}{pf}
\DeclareMathOperator{\Ortho}{O}
\DeclareMathOperator{\diam}{diam}
\DeclareMathOperator{\Gr}{Gr}
\DeclareMathOperator{\GL}{GL}
\DeclareMathOperator{\SO}{SO}
\DeclareMathOperator{\Real}{Re}
\DeclareMathOperator{\Imag}{Im}
\DeclareMathOperator{\dR}{dR}
\DeclareMathOperator{\Ext}{Ext}
\DeclareMathOperator{\Tor}{Tor}

\newcommand{\Proj}{\mathbb{P}}
\newcommand{\RP}{\mathbb{R}P}
\newcommand{\CP}{\mathbb{C}P}

\DeclareMathOperator{\Arg}{Arg}

\newcommand{\C}{\mathbb{C}}

\DeclareMathOperator{\sla}{\mathfrak{sl}}
\newcommand{\norm}[1]{\left\lVert#1\right\rVert}
\newcommand{\transpose}{\intercal}

\newcommand{\conj}[1]{\bar{#1}}
\newcommand{\abs}[1]{\left|#1\right|}

\DeclareMathOperator{\projection}{proj}
\newcommand{\cupp}{\smallsmile}
\author{Jim Fowler}
\course{Math 7851}
\date{Week 5: Cohomology of Grassmannian}

\begin{document}
\maketitle

Last week involved a bit of point-set topology (paracompactness!
normal spaces!) resulting in our identifying bundles over a space $X$
with maps from $X$ into the Grassmannian.  Moreover, identical bundles
result in \textit{homotopic} maps.  This means that, given a bundle
over $X$ and some class $\omega \in H^\star(\mbox{Grassmannian})$, we
can pull it back to get a well-defined class in $H^\star(X)$.  These
\textbf{characteristic classes} are exactly what we have been wanting!
So we had better get started in understanding the cohomology of the
Grassmannian.

The cohomology of the Grassmannian is exactly what you'll find when
you read Chapter~6 and Chapter~7 from Milnor--Stasheff's
\textit{Characteristic Classes}.

Chapter~6 focuses on the cell structure of the Grassmannian and
introduces the \textbf{Schubert cells}. These cells represent a
specific decomposition of the Grassmannian with applications outside
this course. You might already be familiar with \textbf{flags}; the
Schubert cells are certain subsets of the Grassmannian defined by
their incidence with a given flag.  The combinatorics of such gadgets
is important in algebraic geometry.

But for us the main application comes in Chapter~7, where, armed with
this cell structure, we'll compute the cohomology of the Grassmannian.
Earlier we had provided just a list of axioms that we hoped the
Stiefel-Whitney classes would satisfy; now we'll be able to provide
existence and uniqueness proofs.

\end{document}
