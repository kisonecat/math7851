\documentclass{homework}
\usepackage{amsmath}
\DeclareMathOperator{\Mat}{Mat}
\DeclareMathOperator{\End}{End}
\DeclareMathOperator{\Hom}{Hom}
\DeclareMathOperator{\id}{id}
\DeclareMathOperator{\image}{im}
\DeclareMathOperator{\rank}{rank}
\DeclareMathOperator{\nullity}{nullity}
\DeclareMathOperator{\trace}{tr}
\DeclareMathOperator{\Spec}{Spec}
\DeclareMathOperator{\Sym}{Sym}
\DeclareMathOperator{\pf}{pf}
\DeclareMathOperator{\Ortho}{O}
\DeclareMathOperator{\diam}{diam}
\DeclareMathOperator{\Gr}{Gr}
\DeclareMathOperator{\GL}{GL}
\DeclareMathOperator{\SO}{SO}
\DeclareMathOperator{\Real}{Re}
\DeclareMathOperator{\Imag}{Im}
\DeclareMathOperator{\dR}{dR}
\DeclareMathOperator{\Ext}{Ext}
\DeclareMathOperator{\Tor}{Tor}

\newcommand{\Proj}{\mathbb{P}}
\newcommand{\RP}{\mathbb{R}P}
\newcommand{\CP}{\mathbb{C}P}

\DeclareMathOperator{\Arg}{Arg}

\newcommand{\C}{\mathbb{C}}

\DeclareMathOperator{\sla}{\mathfrak{sl}}
\newcommand{\norm}[1]{\left\lVert#1\right\rVert}
\newcommand{\transpose}{\intercal}

\newcommand{\conj}[1]{\bar{#1}}
\newcommand{\abs}[1]{\left|#1\right|}

\DeclareMathOperator{\projection}{proj}
\newcommand{\cupp}{\smallsmile}
\author{Jim Fowler}
\course{Math 7852}
\date{Week 8: Thom isomorphism}

\begin{document}
\maketitle

This is a shortened week because of the Fall Break.

From Milnor--Stasheff's \textit{Characteristic Classes}, read ``The
Thom Isomorphism Theorem.''  It is also worth taking a look at
\textsection 6 of Bott and Tu's \textit{Differential Forms in
  Algebraic Topology}.

The immediate goal this week is to make good on the IOU from earlier
in the semester; when we were building Stiefel-Whitney classes, we
invoked the Thom isomorphism theorem and Steenrod squares.  So this
week, we'll deal with the Thom isomorphism theorem, and next week
we'll reprise some material on the Steenrod squares.

But digging into the Thom isomorphism theorem is itself valuable.
Beyond characteristic classes, our goal this term is to understand
more about manifolds.  A fundamental fact about manifolds is
Poincar\'e duality, so we will spend some time this week thinking
about how the Thom isomorphism theorem is related to this.

\end{document}
