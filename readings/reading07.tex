\documentclass{homework}
\usepackage{amsmath}
\DeclareMathOperator{\Mat}{Mat}
\DeclareMathOperator{\End}{End}
\DeclareMathOperator{\Hom}{Hom}
\DeclareMathOperator{\id}{id}
\DeclareMathOperator{\image}{im}
\DeclareMathOperator{\rank}{rank}
\DeclareMathOperator{\nullity}{nullity}
\DeclareMathOperator{\trace}{tr}
\DeclareMathOperator{\Spec}{Spec}
\DeclareMathOperator{\Sym}{Sym}
\DeclareMathOperator{\pf}{pf}
\DeclareMathOperator{\Ortho}{O}
\DeclareMathOperator{\diam}{diam}
\DeclareMathOperator{\Gr}{Gr}
\DeclareMathOperator{\GL}{GL}
\DeclareMathOperator{\SO}{SO}
\DeclareMathOperator{\Real}{Re}
\DeclareMathOperator{\Imag}{Im}
\DeclareMathOperator{\dR}{dR}
\DeclareMathOperator{\Ext}{Ext}
\DeclareMathOperator{\Tor}{Tor}

\newcommand{\Proj}{\mathbb{P}}
\newcommand{\RP}{\mathbb{R}P}
\newcommand{\CP}{\mathbb{C}P}

\DeclareMathOperator{\Arg}{Arg}

\newcommand{\C}{\mathbb{C}}

\DeclareMathOperator{\sla}{\mathfrak{sl}}
\newcommand{\norm}[1]{\left\lVert#1\right\rVert}
\newcommand{\transpose}{\intercal}

\newcommand{\conj}[1]{\bar{#1}}
\newcommand{\abs}[1]{\left|#1\right|}

\DeclareMathOperator{\projection}{proj}
\newcommand{\cupp}{\smallsmile}
\author{Jim Fowler}
\course{Math 7852}
\date{Week 7: Euler class}

\begin{document}
\maketitle

From Milnor--Stasheff's \textit{Characteristic Classes}, read
Chapter~9, ``Oriented bundles and the Euler class.''  This provides a
segue connecting our $\mathbb{Z}/2\mathbb{Z}$ study to integral
classes by studying a particular cohomology class with integer
coefficients, i.e., the Euler class.  For an \textit{oriented}
rank-$n$ bundle $E$ over $B$, we already have the $n$-th
Stiefel-Whitney class $w_n \in H^n(B;\mathbb{Z}/2\mathbb{Z})$, and you
can think of the Euler class as a refinement of this to a class
$e \in H^n(B;\mathbb{Z})$.  This results in some exciting new
features: the Stiefel-Whitney classes are stable, meaning that
$w_i(E) = w_i(E \oplus \mathbb{R})$ where $\mathbb{R}$ denotes a
trivial line bundle; the Euler class is \textit{unstable}, and indeed
the Euler class of a bundle with a trivial summand vanishes!  Far from
ruining the Euler class, this is truly a feature, because it means
that the Euler class can view unstable phenomena.  For example, the
Euler class can detect the nontriviality of the tangent bundle $TS^2$
even though that bundle is stably trivial.

There are some other references which I think you will find helpful
for this week.  The first is Hatcher's \textit{Vector bundles and
  K-theory}.  From this, read Chapter~3.2 on the Euler class.  I also
want to draw your attention to Bott and Tu's \textit{Differential
  Forms in Algebra Topology} which I think you will find very
inspiring.

\end{document}
