\documentclass{homework}
\usepackage{amsmath}
\DeclareMathOperator{\Mat}{Mat}
\DeclareMathOperator{\End}{End}
\DeclareMathOperator{\Hom}{Hom}
\DeclareMathOperator{\id}{id}
\DeclareMathOperator{\image}{im}
\DeclareMathOperator{\rank}{rank}
\DeclareMathOperator{\nullity}{nullity}
\DeclareMathOperator{\trace}{tr}
\DeclareMathOperator{\Spec}{Spec}
\DeclareMathOperator{\Sym}{Sym}
\DeclareMathOperator{\pf}{pf}
\DeclareMathOperator{\Ortho}{O}
\DeclareMathOperator{\diam}{diam}
\DeclareMathOperator{\Gr}{Gr}
\DeclareMathOperator{\GL}{GL}
\DeclareMathOperator{\SO}{SO}
\DeclareMathOperator{\Real}{Re}
\DeclareMathOperator{\Imag}{Im}
\DeclareMathOperator{\dR}{dR}
\DeclareMathOperator{\Ext}{Ext}
\DeclareMathOperator{\Tor}{Tor}

\newcommand{\Proj}{\mathbb{P}}
\newcommand{\RP}{\mathbb{R}P}
\newcommand{\CP}{\mathbb{C}P}

\DeclareMathOperator{\Arg}{Arg}

\newcommand{\C}{\mathbb{C}}

\DeclareMathOperator{\sla}{\mathfrak{sl}}
\newcommand{\norm}[1]{\left\lVert#1\right\rVert}
\newcommand{\transpose}{\intercal}

\newcommand{\conj}[1]{\bar{#1}}
\newcommand{\abs}[1]{\left|#1\right|}

\DeclareMathOperator{\projection}{proj}
\newcommand{\cupp}{\smallsmile}
\author{Jim Fowler}
\course{Math 7851}
\date{Week 4: Grassmannians}

\begin{document}
\maketitle

After the excitement of last week and the rapid progress we made via
the Stiefel-Whitney classes, we take a step back this week to consider
vector bundles somewhat more ``universally.''  To get started, read
Chapter 5 from Milnor--Stasheff's \textit{Characteristic Classes}.

Our goal this week is to meet the \textbf{Grassmannian}, a space that
parameterizes k- subspaces of a certain dimension of a certain vector
space $V$.  Mathematicians often consider spaces which parameterize
some other geometric object, but the Grassmannian is particularly
important.  Instead of considering the tangent bundle $TM$ over a
smooth manifold $M \subset \mathbb{R}^N$, we can instead consider
sending $x \in M$ to the point in a Grassmannian corresponding to
$T_x M$ as a subspace of $\mathbb{R}^N$.  We reduce the study of
vector bundles to the study of maps into the Grassmannian, which
motivates next week's study of the cohomology of the Grassmannian.
That study, with topics like \textbf{Schubert cells}, involves some
lovely combinatorics.

\end{document}
