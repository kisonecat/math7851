\documentclass{homework}
\usepackage{amsmath}
\DeclareMathOperator{\Mat}{Mat}
\DeclareMathOperator{\End}{End}
\DeclareMathOperator{\Hom}{Hom}
\DeclareMathOperator{\id}{id}
\DeclareMathOperator{\image}{im}
\DeclareMathOperator{\rank}{rank}
\DeclareMathOperator{\nullity}{nullity}
\DeclareMathOperator{\trace}{tr}
\DeclareMathOperator{\Spec}{Spec}
\DeclareMathOperator{\Sym}{Sym}
\DeclareMathOperator{\pf}{pf}
\DeclareMathOperator{\Ortho}{O}
\DeclareMathOperator{\diam}{diam}
\DeclareMathOperator{\Gr}{Gr}
\DeclareMathOperator{\GL}{GL}
\DeclareMathOperator{\SO}{SO}
\DeclareMathOperator{\Real}{Re}
\DeclareMathOperator{\Imag}{Im}
\DeclareMathOperator{\dR}{dR}
\DeclareMathOperator{\Ext}{Ext}
\DeclareMathOperator{\Tor}{Tor}

\newcommand{\Proj}{\mathbb{P}}
\newcommand{\RP}{\mathbb{R}P}
\newcommand{\CP}{\mathbb{C}P}

\DeclareMathOperator{\Arg}{Arg}

\newcommand{\C}{\mathbb{C}}

\DeclareMathOperator{\sla}{\mathfrak{sl}}
\newcommand{\norm}[1]{\left\lVert#1\right\rVert}
\newcommand{\transpose}{\intercal}

\newcommand{\conj}[1]{\bar{#1}}
\newcommand{\abs}[1]{\left|#1\right|}

\DeclareMathOperator{\projection}{proj}
\newcommand{\cupp}{\smallsmile}
\author{Jim Fowler}
\course{Math 7852}
\date{Week 2: Vector bundles}

\begin{document}
\maketitle

This week is about vector bundles.  From Milnor--Stasheff's
\textit{Characteristic Classes}, read both Chapter 2 and Chapter 3.
Chapter~2 introduces vector bundles and provides precise
definitions. Occasionally, the authors cite ``Steenrod's
terminology,'' which will be discussed during our lecture time.  And
anytime we build new objects, we ought to ask about operations on
those objects, so Chapter~3 is all about constructions one can do to
build new vector bundles from old bundles.

Let's step back and reflect on \textit{why} we are studying vector
bundles in the first place! The main focus of this course is on smooth
manifolds, so why bundles?  We have already encountered the tangent
bundle. So one good reason for investigating vector bundles is simply
that they appear in nature---not just in the form of the tangent
bundle, but also in related constructions like the normal
bundle. Understanding bundles associated to smooth manifolds can offer
insight: we will see applications constraining when an
\( n \)-manifold embeds into \( m \)-dimensional space.  Beyond the
tangent and normal bundles, there exist other fun examples of vector
bundles, such as the ``tautological'' bundle over projective space.

Thinking more broadly, vector bundles serve as a way to ``linearize''
the curvy global structure of smooth manifolds. If linear algebra
captures our interest, then surely smoothly varying linear algebra is
even more interesting!  And \textit{sections} of vector bundles pave
the way for defining new functions: these sections can be viewed as a
kind of ``dependent function,'' where the codomain varies based on the
input point, thereby allowing us to study ``twisted'' functions.

\end{document}
