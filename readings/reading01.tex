\documentclass{homework}
\usepackage{amsmath}
\DeclareMathOperator{\Mat}{Mat}
\DeclareMathOperator{\End}{End}
\DeclareMathOperator{\Hom}{Hom}
\DeclareMathOperator{\id}{id}
\DeclareMathOperator{\image}{im}
\DeclareMathOperator{\rank}{rank}
\DeclareMathOperator{\nullity}{nullity}
\DeclareMathOperator{\trace}{tr}
\DeclareMathOperator{\Spec}{Spec}
\DeclareMathOperator{\Sym}{Sym}
\DeclareMathOperator{\pf}{pf}
\DeclareMathOperator{\Ortho}{O}
\DeclareMathOperator{\diam}{diam}
\DeclareMathOperator{\Gr}{Gr}
\DeclareMathOperator{\GL}{GL}
\DeclareMathOperator{\SO}{SO}
\DeclareMathOperator{\Real}{Re}
\DeclareMathOperator{\Imag}{Im}
\DeclareMathOperator{\dR}{dR}
\DeclareMathOperator{\Ext}{Ext}
\DeclareMathOperator{\Tor}{Tor}

\newcommand{\Proj}{\mathbb{P}}
\newcommand{\RP}{\mathbb{R}P}
\newcommand{\CP}{\mathbb{C}P}

\DeclareMathOperator{\Arg}{Arg}

\newcommand{\C}{\mathbb{C}}

\DeclareMathOperator{\sla}{\mathfrak{sl}}
\newcommand{\norm}[1]{\left\lVert#1\right\rVert}
\newcommand{\transpose}{\intercal}

\newcommand{\conj}[1]{\bar{#1}}
\newcommand{\abs}[1]{\left|#1\right|}

\DeclareMathOperator{\projection}{proj}
\newcommand{\cupp}{\smallsmile}
\author{Jim Fowler}
\course{Math 7852}
\date{Week 1: Manifolds}

\begin{document}
\maketitle

Each week, I'll post a few notes on what to read, and I'll try to
\textbf{emphasize} the key topics we plan to cover each week.

For this week, from Milnor--Stasheff's \textit{Characteristic
  Classes}, read the first chapter entitled ``Smooth Manifolds.''
This course is a course in differential topology, and \textbf{smooth
  manifolds} are the central object of our study.  Recall that a
manifold is a topological space that is locally Euclidean but can be
``globally'' more complicated.  A \textit{smooth} manifold is a
topological manifold endowed with a \textbf{smooth structure} --- that
is, a manifold along with some additional data that allows for
differential calculus on that manifold.  For example, we can take
\textbf{derivatives} and we can speak of \textbf{tangent vectors} and
build \textbf{tangent bundles}.

To put all this in some broader context: a topological manifold could
support \textit{different} smooth structures.  Or you may be familiar
with piecewise linear (PL) manifolds, which are topological manifolds
together with a piecewise linear structure on it, i.e., there is an
atlas with with PL transition functions.  A smooth structure gives
rise to a PL structure, but not every PL manifold is smoothable.

Another reference is \textit{Topology from the Differentiable
  Viewpoint} by Milnor, which we'll use to discuss \textbf{regular
  values} and \textbf{critical values}.  We will see later that
\textbf{transversality} generalizes the notion of regular values.

\end{document}
