\documentclass{homework}
\usepackage{amsmath}
\DeclareMathOperator{\Mat}{Mat}
\DeclareMathOperator{\End}{End}
\DeclareMathOperator{\Hom}{Hom}
\DeclareMathOperator{\id}{id}
\DeclareMathOperator{\image}{im}
\DeclareMathOperator{\rank}{rank}
\DeclareMathOperator{\nullity}{nullity}
\DeclareMathOperator{\trace}{tr}
\DeclareMathOperator{\Spec}{Spec}
\DeclareMathOperator{\Sym}{Sym}
\DeclareMathOperator{\pf}{pf}
\DeclareMathOperator{\Ortho}{O}
\DeclareMathOperator{\diam}{diam}
\DeclareMathOperator{\Gr}{Gr}
\DeclareMathOperator{\GL}{GL}
\DeclareMathOperator{\SO}{SO}
\DeclareMathOperator{\Real}{Re}
\DeclareMathOperator{\Imag}{Im}
\DeclareMathOperator{\dR}{dR}
\DeclareMathOperator{\Ext}{Ext}
\DeclareMathOperator{\Tor}{Tor}

\newcommand{\Proj}{\mathbb{P}}
\newcommand{\RP}{\mathbb{R}P}
\newcommand{\CP}{\mathbb{C}P}

\DeclareMathOperator{\Arg}{Arg}

\newcommand{\C}{\mathbb{C}}

\DeclareMathOperator{\sla}{\mathfrak{sl}}
\newcommand{\norm}[1]{\left\lVert#1\right\rVert}
\newcommand{\transpose}{\intercal}

\newcommand{\conj}[1]{\bar{#1}}
\newcommand{\abs}[1]{\left|#1\right|}

\DeclareMathOperator{\projection}{proj}
\newcommand{\cupp}{\smallsmile}
\author{Jim Fowler}
\course{Math 7852}
\date{Week 12: Pontrjagin classes}

\begin{document}
\maketitle

Thus far, we have seen a few examples of ``characteristic classes.''
If we are working modulo two, we have the Stiefel-Whitney classes.  We
have the (unstable!) Euler class.  And if we're working with complex
bundles, we have the Chern classes.  Are there others?

Yes!  Our goal this week is to start the theory of Pontrjagin classes,
which will give us integral classes associated to \textit{real} vector
bundles.  The basic idea is straightforward: if you want an invariant
of real bundles, just complexify your bundle and then use Chern
classes!

These Pontrjagin classes satisfy some nice properties, like stability.
Having thrown away the odd Chern classes of the complexification, we
don't have quite so nice a product formula.  Of course, as is usual in
mathematics, it's these imperfect analogies that make the subject so
interesting.  We'll get started by reading the ``Pontrjagin classes''
chapter from Milnor--Stasheff's \textit{Characteristic Classes}.

\end{document}
