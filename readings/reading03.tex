\documentclass{homework}
\input{preamble}
\author{Jim Fowler}
\course{Math 7852}
\date{Week 3: Stiefel-Whitney classes}


\begin{document}
\maketitle

From Milnor--Stasheff's \textit{Characteristic Classes}, read
Chatper~4 which introduces \textbf{Stiefel-Whitney classes} from an
\textit{axiomatic} perspective.  It won't be until later in this
course when we actually show that cohomology classes satisfying these
axioms exist.

In the broadest perspective, Stiefel-Whitney classes are our first
example of \textbf{characteristic classes}, which are certain
cohomology classes associated with a vector bundle. These cohomology
classes quantify how ``twisted'' the vector bundle is; for a trivial
product, the cohomology classes are trivial. Part of the interest in
characteristic classes is that they are a common theme tying together
algebraic topology, algebraic geometry, and differential geometry.

The Stiefel-Whitney classes specifically associate to a bundle
\( E \to B \) certain classes
\( w_i \in H^i(B; \mathbb{Z}/2\mathbb{Z}) \). These classes encode
features we care about: the first Stiefel-Whitney class \( w_1 \) is
trivial if and only if the bundle is orientable. But be warned: these
classes aren't complete invariants of bundles. For example, \( TS^2 \)
is not trivial, but \( TS^2 \) is stably trivial, meaning that
\( TS^2 \oplus \mathbb{R} \) is a trivial bundle, and the
Stiefel-Whitney classes vanish for stably trivial bundles.

\end{document}
